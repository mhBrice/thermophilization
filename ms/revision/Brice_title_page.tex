\PassOptionsToPackage{unicode=true}{hyperref} % options for packages loaded elsewhere
\PassOptionsToPackage{hyphens}{url}
%
\documentclass[a4paperpaper,]{article}
\usepackage{lmodern}
\usepackage{amssymb,amsmath}
\usepackage{ifxetex,ifluatex}
\usepackage{fixltx2e} % provides \textsubscript
\ifnum 0\ifxetex 1\fi\ifluatex 1\fi=0 % if pdftex
  \usepackage[T1]{fontenc}
  \usepackage[utf8]{inputenc}
  \usepackage{textcomp} % provides euro and other symbols
\else % if luatex or xelatex
  \usepackage{unicode-math}
  \defaultfontfeatures{Ligatures=TeX,Scale=MatchLowercase}
\fi
% use upquote if available, for straight quotes in verbatim environments
\IfFileExists{upquote.sty}{\usepackage{upquote}}{}
% use microtype if available
\IfFileExists{microtype.sty}{%
\usepackage[]{microtype}
\UseMicrotypeSet[protrusion]{basicmath} % disable protrusion for tt fonts
}{}
\IfFileExists{parskip.sty}{%
\usepackage{parskip}
}{% else
\setlength{\parindent}{0pt}
\setlength{\parskip}{6pt plus 2pt minus 1pt}
}
\usepackage{hyperref}
\hypersetup{
            pdfborder={0 0 0},
            breaklinks=true}
\urlstyle{same}  % don't use monospace font for urls
\usepackage[margin=1in]{geometry}
\setlength{\emergencystretch}{3em}  % prevent overfull lines
\providecommand{\tightlist}{%
  \setlength{\itemsep}{0pt}\setlength{\parskip}{0pt}}
\setcounter{secnumdepth}{0}
% Redefines (sub)paragraphs to behave more like sections
\ifx\paragraph\undefined\else
\let\oldparagraph\paragraph
\renewcommand{\paragraph}[1]{\oldparagraph{#1}\mbox{}}
\fi
\ifx\subparagraph\undefined\else
\let\oldsubparagraph\subparagraph
\renewcommand{\subparagraph}[1]{\oldsubparagraph{#1}\mbox{}}
\fi

% set default figure placement to htbp
\makeatletter
\def\fps@figure{htbp}
\makeatother

\usepackage{setspace}
\setstretch{1,5}
\usepackage{lineno}
\linenumbers
\raggedright

\date{}

\begin{document}

\hypertarget{disturbances-amplify-tree-community-responses-to-climate-change-in-the-temperate-boreal-ecotone}{%
\section{Disturbances amplify tree community responses to climate change
in the temperate-boreal
ecotone}\label{disturbances-amplify-tree-community-responses-to-climate-change-in-the-temperate-boreal-ecotone}}

Marie-Hélène Brice* \textsuperscript{1,2}

Kevin Cazelles \textsuperscript{3}

Pierre Legendre \textsuperscript{1,2}

Marie-Josée Fortin \textsuperscript{4}

\hypertarget{institutional-affiliations}{%
\subsection{Institutional
affiliations}\label{institutional-affiliations}}

\begin{enumerate}
\def\labelenumi{\arabic{enumi}.}
\tightlist
\item
  Département de sciences biologiques, Université de Montréal, C.P.
  6128, succursale Centre-ville, Montréal, QC, Canada H3C 3J7.
\item
  Québec Centre for Biodiversity Science, McGill University, Montréal,
  QC, Canada H3A 1B1.
\item
  Department of Integrative Biology, University Of Guelph, Guelph, ON,
  Canada N1G 2W1
\item
  Department of Ecology and Evolutionary Biology, University of Toronto,
  Toronto, ON, Canada M5S 3B2.
\end{enumerate}

Corresponding author * : marie-helene.brice@umontreal.ca

\hypertarget{acknowledgements}{%
\subsection{Acknowledgements}\label{acknowledgements}}

We thank Steve Vissault for helping us with the forest dataset, and
Daniel W. McKenney and Pia Papadopol for help with the climate data. We
are also grateful to Pierre Grondin, Aurélie Chalumeau and Marie-Claude
Lambert, as well as one anonymous reviewer for providing many helpful
suggestions and comments that improved our manuscript. This research was
supported by Natural Sciences and Engineering Research Council of Canada
(NSERC) research grant no. 7738 to PL and no. 5134 to MJF.

\hypertarget{biosketch}{%
\subsection{Biosketch}\label{biosketch}}

Marie-Hélène Brice is a Ph.D.~candidate at Université de Montréal. She
is interested in the natural and anthropogenic processes that structure
biodiversity spatially and temporally, with a special interest for
vegetation. She uses taxonomic and functional approaches to understand
the link between the ongoing global change and community reorganization.

\end{document}
